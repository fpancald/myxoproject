\documentclass[12pt]{article}
\usepackage{amsmath}
\usepackage{amsfonts}
\usepackage{amssymb}
\usepackage{amsthm}
\topmargin-.5in
\textwidth6.5in
\textheight9in
\oddsidemargin0in
\evensidemargin0in

\begin{document}
\hfill {\large Project: }Mathematical Model of Polarity in Myxobacteria Xanthus
\vspace{9mm}
\section*{ \centerline{April 13, 2015}
\centerline{ \bf \large Proposed model}
\centerline{ \bf \large report author: Francesco Pancaldi}}

\begin{enumerate}
\item RomR protein can exist in multiple forms (e.g. monomer and dimer or other polymers or active and inactive states) and each form has specific diffusion coefficient, receptor attachement and detachment rates;
\item RomR can change with certain rates (either constants or dependent on other variables) from one form to an other;
\item only some RomR forms can bound with RomR-receptors on the cell body;
\item RomR receptors are present only at the poles of the cell (except during division);
\item RomR receptors can be either active or non-active and only active receptors can bind RomR;
\item RomR receptors are activated (or deactivated) by a protein that diffuses in the cell body and reacts with other proteins simillarly to the MinCDE system in E. Coli;
\item during division new receptors are produced at the septum location due to the formation of the new poles;
\item during division the two daughter cells can exchange protein for a certain ammount of time (either freely or with certian limitations) then the communication is interrupted; 
\end{enumerate}
Therefore we will propose the following equations for a 1D PDE model:\\ \\
Equation for i-th form of RomR when not bounded to the membrane (note:$C_i=C_i(x,t)$).
\begin{equation}
\frac{\partial C}{\partial t}=\frac{\partial}{\partial x}(D\frac{\partial C}{\partial x})- r(n-c)C+R c
\end{equation}
where $D_i$ is the diffusion coefficient (could be space dependent), $F_i^{in}$ and $F_i^{out}$ are the appropriate influx and outflux functions due to conversion between forms, $r_i$ and $R_i$ are the binding and unbinding rates to the RomR-receptors.\\ \\
Equation for i-th form of RomR when bounded to the membrane (note:$c_i=c_i(x,t)$).
\begin{equation}
\frac{\partial c}{\partial t}=r(n-c)C+R c
\end{equation}
Equation for density of active receptors.
\begin{equation}
   n(x,t) = \left\{
     \begin{array}{lcr}
       f (N, k, h),& x\in [0, l_1]\cap [L-l_1, L] & \forall t\\
 g(t)f (N, k, h) ,& x \in[L/2-l_2, L/2 +l_2 ] & t \geq T_1

     \end{array}
   \right.
\end{equation}
where $N$ is the maximum density of active receptors at any given location, the delta indicate space or time constarints (if we are at the poles or center and if division started), $f$ is a function quantifying the ammount of active receptors dependent on $N$ and the quantity of chemicals $k, K$, and $g$ is a function for the growth of new receptors at the center.
The variables $k, K$ are the densities for unbounded and bounded form of a chemical that interfers with the receptors and interacts with a second chemical $h,H$ as described by the following equations:
\begin{equation}
\frac{\partial K}{\partial t}=\frac{\partial}{\partial x}(D_K\frac{\partial K}{\partial x})-\frac{\sigma_1K}{1+\sigma_1'h}+\sigma_2hk
\end{equation}
\begin{equation}
\frac{\partial k}{\partial t}=\frac{\sigma_1K}{1+\sigma_1'h}-\sigma_2hk
\end{equation}
\begin{equation}
\frac{\partial H}{\partial t}=\frac{\partial}{\partial x}(D_H\frac{\partial H}{\partial x})+\frac{\sigma_4h}{1+\sigma_4'K}-\sigma_3KH
\end{equation}
\begin{equation}
\frac{\partial H}{\partial t}=-\frac{\sigma_4h}{1+\sigma_4'K}+\sigma_3KH
\end{equation}
(these equations are taken from the literature about the MinCDE system and have been shown to produce oscillation from on pole to the other with a period T dependent on the parameters).
\end{document}
